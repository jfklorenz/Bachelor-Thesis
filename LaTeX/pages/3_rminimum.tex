%%%%%%%%%%%%%%%%%%%%%%%%%%%%%%%%%%%%%%%%%%%%%%%%%%%%%%%%%%%%%%%%%
\section{\Rm}	%	RMINIMUM
%%%%%%%%%%%%%%%%%%%%%%%%%%%%%%%%%%%%%%%%%%%%%%%%%%%%%%%%%%%%%%%%%


\noindent
Der im Paper~\cite{meyer1} vorgestellte Algorithmus \Rm ist ein randomisierter rekursiver Algorithmus zum Auffinden des kleinsten Elements einer Menge $X$ von $n$ verschiedenen Elementen.\\[.05cm]
Er erhält als Eingabe das $2$-Tupel $(X, k)$, bestehend aus einer Menge $X:=\{x_1,\cdots,x_n\}$ mit strenger Totalordnung und einem \tuning $k(n)$, der den \trade zwischen der \fg \fgm des kleinsten Elementes und der maximal erwarteten \fg \fgr der übrigen Elemente regelt. Für \fgpair ergeben sich je nach Wahl des Parameters $k(n)$ spezifische Wertepaare.\\[0.05cm]
Hierfür wird im Laufe dieses Kapitels gewisse Abschätzungen vorstellt und diese abschließend experimentell untersucht.


%%%%%%%%%%%%%%%%%%%%%%%%%%%%%%%%%%%%%%%%%%%%%%%%%%%%%%%%%%%%%%%%%
