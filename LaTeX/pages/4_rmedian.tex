%%%%%%%%%%%%%%%%%%%%%%%%%%%%%%%%%%%%%%%%%%%%%%%%%%%%%%%%%%%%%%%%%
\section{\RM}	%	RMEDIAN
%%%%%%%%%%%%%%%%%%%%%%%%%%%%%%%%%%%%%%%%%%%%%%%%%%%%%%%%%%%%%%%%%

Der Algorithmus \RM ist ein in Erwartung Praxis-optimaler Algorithmus zum Auffinden des Median Elements. Hierbei wird explizit auf die im Paper~\cite{meyer2} vorgestellte Version Bezug genommen.\\[.1cm]
Mit Hilfe eines Parameters $k(n)$ kann man den \trade zwischen der erwarteten \fg \fgM des Median Elements und der maximal erwarteten \fg \fgr aller übrigen Elemente kontrollieren. Dies erlaubt je nach Adjustierung des Parameters verschiedene Werte für das Paar \fgpair zwischen\\
$\langle\mathcal{O}(\log\log(n)),\mathcal{O}(\sqrt{n})\rangle$ und $\langle\mathcal{O}(\log(n)/\log\log(n)),\mathcal{O}(\log(n))\rangle$.

%%%%%%%%%%%%%%%%%%%%%%%%%%%%%%%%%%%%%%%%%%%%%%%%%%%%%%%%%%%%%%%%%