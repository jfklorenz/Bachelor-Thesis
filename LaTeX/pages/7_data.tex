%%%%%%%%%%%%%%%%%%%%%%%%%%%%%%%%%%%%%%%%%%%%%%%%%%%%%%%%%%%%%%%%%
\section{Data}		%	DATA
%%%%%%%%%%%%%%%%%%%%%%%%%%%%%%%%%%%%%%%%%%%%%%%%%%%%%%%%%%%%%%%%%

%%%%%%%%%%%%%%%%%%%%%%%%%%%%%%%%%%%%%%%%%%%%%%%%%%%%%%%%%%%%%%%%%
Die Grafiken dieser Arbeit beruhen auf den durch die Implementierung experimentell gewonnen Daten, welche in vollem Umfang bei bei \href{https://github.com/jfklorenz}{Git Hub} eingesehen werden können.\\[.05cm]
Zur Veranschaulichung werden hier die jeweiligen Mittelwerte der experimentellen Daten bezüglich der nach Theorem~\ref{theo: min_5} geforderten Parametrisierung von $k(n)$ aufgeführt.\\[.05cm]
Die tabellarischen Werte entsprechen jeweils dem Mittelwert der Daten aus zehntausend Wiederholungen bezüglich der Auf- und Abrundung des Parameters $k(n)$ sowie der in Kapitel~\ref{other: weigh} diskutierten Gewichtung.

%%%%%%%%%%%%%%%%%%%%%%%%%%%%%%%%%%%%%%%%%%%%%%%%%%%%%%%%%%%%%%%%%
%	DATA - RMINIMUM - Case 2 : log(n) / loglog(n)
%%%%%%%%%%%%%%%%%%%%%%%%%%%%%%%%%%%%%%%%%%%%%%%%%%%%%%%%%%%%%%%%%

\begin{center}
\begin{tabular}{|c|c||c|c|c||c|c|c||c|c|c|}
        	\hline
        	\multicolumn{11}{|c|}{\ccb \textbf{Mittelwerte für $k(n) = \log_2(n) / \log_2\log_2(n)$}}\\
        	\hline
        	\hline
        	\multicolumn{1}{|c|}{\cca \textbf{$\log_2(n)$}}&
        	\multicolumn{1}{c|}{\cca \textbf{$k$}}&
        	\multicolumn{1}{c|}{\cca \textbf{$\lfloor k\rfloor$}}&
        	\multicolumn{1}{c|}{\cca \textbf{\fgm}}&
        	\multicolumn{1}{c|}{\cca \textbf{\fgr}}&
        	\multicolumn{1}{c|}{\cca \textbf{$\lceil k\rceil$}}&
        	\multicolumn{1}{c|}{\cca \textbf{\fgm}}&
        	\multicolumn{1}{c|}{\cca \textbf{\fgr}}&
        	\multicolumn{1}{c|}{\cca \textbf{Ratio}}&
        	\multicolumn{1}{c|}{\cca \textbf{\fgm}}&
        	\multicolumn{1}{c|}{\cca \textbf{\fgr}}\\
        	\hline
        	$6$&$2.32$&$2$& $6.63$&$6.41$ &$3$& $6.50$&$6.32$ &$68-32$&$6.59$ &$6.38$  \\
            \hline
            $7$&$2.49$&$2$&$7.75$ &$7.56$ &$3$& $7.51$ &$7.33$ &$51-49$&$7.61$ &$7.42$  \\
            \hline
            $8$&$2.67$&$2$&$9.11$ &$9.11$ &$3$&$8.84$ &$8.79$ &$33-67$&$8.88$ & $8.84$ \\
            \hline
            $9$&$2.84$&$2$&$10.64$ &$10.56$ &$3$&$10.10$ &$10.03$ &$16-84$&$10.60$ &$10.51$  \\
            \hline
            $10$&$3.01$&$3$&$11.25$ &$11.20$ &$4$&$10.87$ &$11.04$ &$99-1$&$11.12$ &$11.14$  \\
            \hline
            $11$&$3.18$&$3$&$12.65$ &$12.55$ &$4$&$12.05$&$12.50$ &$82-18$&$12.30$ &$12.52$  \\
            \hline
            $12$&$3.35$&$3$&$13.78$ &$13.66$ &$4$&$13.19$&$13.18$  &$65-35$&$13.29$ &$13.26$  \\
            \hline
            $13$&$3.51$&$3$&$15.01$ &$15.04$ &$4$&$14.45$ &$14.95$  &$49-51$&$14.97$ &$15.04$  \\
            \hline
            $14$&$3.68$&$3$&$16.43$ &$16.33$ &$4$& $15.51$ &$15.43$  &$32-68$&$16.15$ &$16.05$  \\
            \hline
            $15$&$3.84$&$3$&$17.53$ &$17.39$ &$4$& $16.75$ &$17.01$  &$16-84$&$17.11$ &$17.19$  \\
            \hline
            $16$&$4$&$4$&$17.89$ &$18.14$ &$4$& $17.89$ &$18.14$  &$any$&$17.89$ &$18.14$  \\
            \hline
            $17$&$4.16$&$4$&$18.36$ &$20.77$ &$5$& $19.03$ &$19.11$  &$84-16$&$18.36$ &$20.77$  \\
            \hline
            $18$&$4.32$&$4$&$19.40$ &$21.04$ &$5$& $20.30$  &$20.83$  &$68-32$&$19.61$ &$20.99$  \\
            \hline
            $19$&$4.47$&$4$&$20.55$ &$22.96$ &$5$&  $21.37$  &$21.30$  &$53-47$&$20.92$ &$22.21$  \\
            \hline
            $20$&$4.63$&$4$&$21.6$ &$23.20$ &$5$&$22.63$  &$23.00$  &$37-63$&$22.30$ &$23.06$  \\
            \hline     
        \end{tabular}\label{Tab: RMin - k=1/2}
        \captionof{table}{Data : \Rm - $k(n) = \log_2(n) / \log_2\log_2(n)$}% Add 'table' caption
\end{center}

%%%%%%%%%%%%%%%%%%%%%%%%%%%%%%%%%%%%%%%%%%%%%%%%%%%%%%%%%%%%%%%%%
























