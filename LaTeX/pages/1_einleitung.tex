%%%%%%%%%%%%%%%%%%%%%%%%%%%%%%%%%%%%%%%%%%%%%%%%%%%%%%%%%%%%%%%%%
\section{Einleitung}	%	THEORIE
%%%%%%%%%%%%%%%%%%%%%%%%%%%%%%%%%%%%%%%%%%%%%%%%%%%%%%%%%%%%%%%%%

\noindent
Das Thema dieser Bachelorarbeit ist die Implementierung der Algorithmen \Rm und \RM aus den am 14. Juni 2018 und 9. Januar 2019 publizierten Versionen des Papers \textit{Fragile Complexity of Comparison-Based Algorithms}~\cite{meyer1}, sowie eine Evaluation der experimentell ausgewerteten Daten.\\[.1cm]
Der Begriff der \fg genannten Klassifizierung von vergleichsbasierten Sortieralgorithmen bezüglich der maximalen Teilnahme ihrer Elemente ist ein erst vor kurzem in das Interesse der wissenschaftlichen Forschung gerückt. 
Dementsprechend existiert ein konkreter Bedarf an empirischen Daten, die man gegen die theoretisch entwickelten Schranken abgleichen kann.\\[0.05cm]
Zu Beginn dieser Arbeit wird der Begriff der \fg sowie weitere benötigte Definitionen und Begrifflichkeiten eingeführt. Im Anschluss wird die Implementierung der Algorithmen sowie die Auswahl der untersuchenden Schranken besprochen. Entsprechend dieser Wahl wird eine sinnvolle Parametrisierung der Eingabe diskutiert.\\[.05cm]
Im Folgenden werden die verwendeten Analysemethoden vorgestellt. Das primär genutzte Werkzeug ist nicht-linearen Regression, die mithilfe des \textit{Marquardt-Levenberg}-Algorithmus~\cite{gnu2} realisiert wurde. Dieser nutzt die Methode der kleinsten Quadrate, um eine Kurve durch die gewonnene Datenwolke zu berechnen.\\[.05cm]
Abschließend werden die Resultate der Analyse in tabellarischer sowie grafischer Form präsentiert. Für den Algorithmus \Rm konnten alle empirisch getesteten Schranken eindeutig bekräftigt werden. Für den Algorithmus \RM hingegen konnten nicht genug Experimentaldaten gesammelt werden, um eine Aussage fest unterstützen zu können. Im Verlauf der Arbeit hat sich jedoch auch kein merklich Widerspruch zu den im Paper diskutierten Schranken ergeben, so dass es ein positiver Ansatz für weitere Forschung ist.




%	Gliederung


%	Ziel -> wo geht es hin?



%	Gliedert sich in bisherigen Forschungsstand ein?

%	Wie gehts du methodisch vor