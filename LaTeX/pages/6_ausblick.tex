%%%%%%%%%%%%%%%%%%%%%%%%%%%%%%%%%%%%%%%%%%%%%%%%%%%%%%%%%%%%%%%%%
\section{Ausblick}		%	AUSBLICK
%%%%%%%%%%%%%%%%%%%%%%%%%%%%%%%%%%%%%%%%%%%%%%%%%%%%%%%%%%%%%%%%%

\noindent
Im Verlaufe dieser Arbeit konnten die in den Theoremen~\ref{theo: min_3}, ~\ref{theo: min_3} und ~\ref{theo: min_3} der Version~\cite{meyer1} aufgestellten Schranken bezüglich der \fg der Elemente des Algorithmus \Rm eindeutig bestärkt werden. Auch wenn weitere Testfälle denkbar wären, so ist die Korrelation zwischen den experimentell gesammelten Daten bereits jetzt unverkennbar.\\[.05cm]
Für den Algorithmus \RM konnte die Linearität der Arbeit $w(n)$ und somit auch Theorem~\ref{theo: med_30} eindeutig bestätigt werden. Für die \fg des Medians sowie aller übrigen Elemente scheinen die Schranken für alle bisher angesammelten Daten eine gute Abschätzung zu liefern, so dass die positive Erwartung besteht, auch die Theoreme~\ref{theo: med_28} und ~\ref{theo: med_29} aus Paper~\cite{meyer2} in naher Zukunft durch weitere Experimentalanalyse weiter stützen zu können.\\[.05cm]
Von weiterem wissenschaftlichen Interesse wäre hier die Entwicklung einer bezüglich ihrer Laufzeit optimierten Implementierung, da die Menge an experimentell gesammelten Daten entscheidend für die Qualität der Analyse ist.\\[.2cm]



%%%%%%%%%%%%%%%%%%%%%%%%%%%%%%%%%%%%%%%%%%%%%%%%%%%%%%%%%%%%%%%%%



