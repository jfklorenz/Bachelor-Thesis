%%%%%%%%%%%%%%%%%%%%%%%%%%%%%%%%%%%%%%%%%%%%%%%%%%%%%%%%%%%%%%%%%
\label{New Commands}	%	NEW COMMANDS
%%%%%%%%%%%%%%%%%%%%%%%%%%%%%%%%%%%%%%%%%%%%%%%%%%%%%%%%%%%%%%%%%
\label{New Commands : Basic}
%	Array Stretch
\renewcommand{\arraystretch}{1.3}

%	Other
\newcommand\mycommfont[1]{\footnotesize\ttfamily\textcolor{blue}{#1}}
\SetCommentSty{mycommfont}

%	Definition, Theorem, Korollar, Lemma
\newtheorem{definition}{Definition}[section]
\newtheorem{theorem}{Theorem}[section]
\newtheorem{corollary}{Corollary}[theorem]
\newtheorem{lemma}[theorem]{Lemma}

%%%%%%%%%%%%%%%%%%%%%%%%%%%%%%%%%%%%%%%%%%%%%%%%%%%%%%%%%%%%%%%%%
%	Math
\label{New Commands : Math}
\newcommand{\cm}{\textcolor{greengreen}{\text{\ding{51}}}}	% Green Checkmark
\newcommand{\xm}{\textcolor{red}{\text{\ding{55}}}}	% Red Xmark
\newcommand{\mN}{\mathbb{N}}	% \mathbb{N}
\newcommand{\mR}{\mathbb{R}}	% \mathbb{R}
\newcommand{\mQ}{\mathbb{Q}}	% \mathbb{Q}
\newcommand{\mC}{\mathbb{C}}	% \mathbb{C}
\newcommand{\mbb}[1]{\mathbb{#1}}	% \mathbb{X}
\newcommand{\mcal}[1]{\mathcal{#1}}	% \mathcal{X}

\label{New Commands : Math - Functions}
\newcommand{\mset}[1]{\left\{#1\right\}}			% Menge Input
\newcommand{\mabs}[1]{\left|#1\right|}				% Betrag Input
\newcommand{\mpoly}[3]{#1_{#3}#2^{#3} + \ldots + #1_1#2 + #1_0}	% Polynom Input
\newcommand{\mto}[2]{#1,\ldots ,#2}					% x,...,y Input
\newcommand{\minto}[2]{\in\{ #1,\ldots ,#2\}}		% \in\{x,...,y\}
\newcommand{\msum}[3]{\sum\limits_{i=#1}^{#2} #3} 	% Summe Input
\newcommand*{\mSum}[3]{\sum\limits_{i=#1}^{#2} #3} 	% Summe Input
\newcommand{\mlim}[3]{\lim\limits_{#1\rightarrow #2} #3}	% Limes Input
\newcommand*{\mLim}[3]{\lim\limits_{#1\rightarrow #2} #3}	% Limes Input

%	QED
\newcommand*{\qqed}{\hfill\ensuremath{\square}}	% QED
\newcommand*{\qedb}{\hfill\ensuremath{\blacksquare}}	% QED

%	Text above = and ~
\newcommand{\eqtxt}[1]{\ensuremath{\stackrel{\text{#1}}{=}}}
\newcommand{\aptxt}[1]{\ensuremath{\stackrel{\text{#1}}{\approx}}}

\label{New Commands : Laufzeit}
\newcommand{\mO}{\mathcal{O}}	% \mathcal{O}
\newcommand{\mOf}[1]{\mathcal{O}(#1)}
\newcommand{\mOn}{\mathcal{O}(n)}	% \mathcal{O}
\newcommand{\mOnn}{\mathcal{O}(n^2)}	% \mathcal{O}
\newcommand{\mOnnn}{\mathcal{O}(n^3)}	% \mathcal{O}
\newcommand{\mOl}{\mathcal{O}(\log (n))}	% \mathcal{O}
\newcommand{\mOnl}{\mathcal{O}(n\cdot \log (n))}	% \mathcal{O}
\newcommand{\mT}{\Theta}	% \Theta
\newcommand{\mOm}{\Omega}	% \Omega
\newcommand{\mve}{\varepsilon}
\newcommand{\mOi}[1]{\ensuremath{\mathop{}\mathopen{}\mathcal{O}\mathopen{}\left(#1\right)}}	% \mOi{X} -> \mathcal{O}(X)
\newcommand{\Ra}{\Rightarrow }
\newcommand{\ra}{\rightarrow }

\label{New Commands : Brackets}
\newcommand{\mbraa}[1]{\left(#1\right)} 	% encloses the argument using stretchable parentheses
\newcommand{\mbrab}[1]{\left[#1\right]} 	% encloses the argument using stretchable square brackets
\newcommand{\mbrac}[1]{\left\{#1\right\}}	% encloses the argument using stretchable curved brackets

%%%%%%%%%%%%%%%%%%%%%%%%%%%%%%%%%%%%%%%%%%%%%%%%%%%%%%%%%%%%%%%%%
%	Text
\label{New Commands : Rules}
\newcommand{\myRule}[3][black]{\textcolor{#1}{\rule[2mm]{#2}{#3}}}
\renewcommand{\dateseparator}{ -- }
\newcommand{\nth}[1]{\begin{large}\textbf{#1}\end{large}\newline\noindent\myRule[gray]{\textwidth}{1pt}}

\label{New Commands : Spaces}
\newcommand{\va}{\vspace*{.1cm}}
\newcommand{\vb}{\vspace*{.2cm}}
\newcommand{\vc}{\vspace*{.3cm}}
\newcommand{\vd}{\vspace*{.5cm}}
\newcommand{\np}{\newpage}

\label{New Commands : Tabular}
\newcommand{\cca}{{\cellcolor{gray2!40!white}}}			% Subtitle
\newcommand{\ccb}{{\cellcolor{gray2!60!white}}}			% Title
\newcommand{\ccp}{{\cellcolor{greengreen!60!white}}}	% Positive Green
\newcommand{\ccn}{{\cellcolor{redred!60!white}}}		% Negative Red
\newcommand{\ccz}[1]{{\cellcolor{#1}}}					% Input
\newcommand{\tc}[2]{\begin{center}\begin{tabular}{#1}#2\end{tabular}\end{center}}	% centered tabular

\newcommand{\mMp}[1]{\begin{pmatrix} #1\end{pmatrix}}
\newcommand{\mMs}[1]{\bigl(\begin{smallmatrix} #1\end{smallmatrix}\bigr)}

\label{New Commands : Comments}
\newcommand{\ccc}[1]{}
\newcommand{\cccq}[1]{\textit{#1} }
\newcommand{\cccp}[1]{\textbf{#1} }
\newcommand{\cccz}[1]{\noindent\textbf{Zusammenfassung/Fragen:}\\ \noindent\texttt{#1}\\[.2cm]}

%%%%%%%%%%%%%%%%%%%%%%%%%%%%%%%%%%%%%%%%%%%%%%%%%%%%%%%%%%%%%%%%%
\label{New Commands : Bachelor-Thesis}
%	Text
\newcommand{\Rm}{\textsc{RMinimum} }
\newcommand{\RM}{\textsc{RMedian} }
\newcommand{\mE}{\textit{kleinste Element} }
\newcommand{\nmE}{\textit{nicht-minimum Elemente} }
\newcommand{\fg}{\textit{Fragile Complexity} }
\newcommand{\wk}{\textit{Arbeit} }
\newcommand{\tuning}{\textit{Tuning Parameter} }
\newcommand{\trade}{\textit{Trade-Off} }
\newcommand{\sampling}{\textit{Sampling-Phase} }
\newcommand{\probing}{\textit{Probing-Phase} }
\newcommand{\minele}{\textit{Minimum-Element} }
\newcommand{\menwin}{\textit{Gewinner-Menge} }
\newcommand{\menloss}{\textit{Verlierer-Menge} }
\newcommand{\alg}{Algorithmus }
\newcommand{\jp}{\textit{Jupyter} }
\newcommand{\tf}{\textit{Testfall} }
\newcommand{\csv}{\textit{.csv}-Dateien }

%	Formula
\newcommand{\fgm}{$f_{min}(n)$ }
\newcommand{\fgr}{$f_{rem}(n)$ }
\newcommand{\fgM}{$f_{med}(n)$ }
\newcommand{\mfgm}{f_{min}(n)}
\newcommand{\mfgr}{f_{rem}(n)}
\newcommand{\mfgM}{f_{med}(n)}
\newcommand{\mfge}{f_{e}(n)}
\newcommand{\mfgc}{f(n)}

%	Complex Formula
\newcommand{\fgpair}{$\langle \mathbb{E}[f_{min}(n)],\max \mathbb{E}[f_{rem}(n)]\rangle$ }
\newcommand{\elog}{\varepsilon^{-1}\log(\log(n))}

%%%%%%%%%%%%%%%%%%%%%%%%%%%%%%%%%%%%%%%%%%%%%%%%%%%%%%%%%%%%%%%%%

